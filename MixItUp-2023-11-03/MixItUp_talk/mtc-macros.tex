\usepackage[absolute, overlay]{textpos}

% Define a new environment for a code snippet box
\newtcolorbox{codebox}[1][]{
  colback=white,
  colframe=black!75!black,
  width=\linewidth,
  arc=4pt,
  outer arc=4pt,
  left=6pt,
  right=6pt,
  boxsep=5pt,
  #1
}

\newtcolorbox{fancycode}[3][]{
  colback=white,
  colframe=black!75!black,
  width=0.8\linewidth,
  arc=1pt,
  outer arc=1pt,
  left=1pt,
  right=1pt,
  top=1pt,
  bottom=1pt,
  boxsep=0pt,
  overlay={
    % Draw the tab
    \draw[fill=white, rounded corners=1mm] 
      ([xshift=-0.5cm, yshift=2pt]frame.north east) --
      ([yshift=2pt]frame.north east) --
      (frame.north east) --
      ([xshift=-0.5cm]frame.north east) -- cycle;
    % Add the source link
    \node[anchor=north east, font=\tiny, text=black] 
      at ([xshift=-0.25cm, yshift=1pt]frame.north east) {Source: \href{#2}{\texttt{#3}}};
  },
  #1
}

% Define a new tcolorbox environment for the code block
\newtcolorbox{codeblock}[3][]{
  colback=white,
  colframe=black!75!black,
  width=0.8\linewidth,
  arc=1pt,
  outer arc=1pt,
  left=1pt,
  top=1pt,
  bottom=1pt,
  right=1pt,
  boxsep=0pt,
  overlay={
    \node[anchor=south east, font=\tiny, text=black] at (frame.south east) {Source: \href{#2}{\texttt{#3}}};
  },
  #1
}

\newtcolorbox{emptyblock}[2][]{
  colback=white,
  colframe=black!75!black,
  width=0.8\linewidth,
  arc=1pt,
  outer arc=1pt,
  left=1pt,
  top=1pt,
  bottom=1pt,
  right=1pt,
  boxsep=0pt,
  overlay={
    % Draw the tab
    \draw[fill=white, draw=black!75!black, rounded corners=1mm] 
      ([xshift=-1cm, yshift=1cm]frame.north east) rectangle ++(1cm, 0.5cm);
    % Add the text
    \node[anchor=north east, font=\tiny, text=black] 
      at ([xshift=-0.5cm, yshift=1.25cm]frame.north east) {#2};
  },
  #1
}
\newcommand{\blockwithtab}[2]{
  \begin{tikzpicture}
    % Draw the box
    \node[anchor=north west, draw=black!75!black, fill=white, rounded corners=1mm, text width=0.8\linewidth, inner sep=5pt] (box) {#1};
    % Draw the tab
    \node[anchor=south east, draw=black!75!black, fill=white, rounded corners={1mm,1mm,0mm,0mm}, text width=1cm, inner sep=2pt, yshift=0.5pt] at (box.north east) {#2};
  \end{tikzpicture}
}

%\newcommand{\blockwithtab}[2]{
%  \begin{tikzpicture}
%    % Draw the box
%    \node[anchor=north west, draw=black!75!black, fill=white, rounded corners=1mm, text width=0.8\linewidth, inner sep=5pt] (box) {#1};
%    % Draw the tab
%    \node[anchor=south, draw=black!75!black, fill=white, rounded corners=1mm, text width=1cm, inner sep=2pt, yshift=2pt] at (box.north east) {#2};
%  \end{tikzpicture}
%}
%\newcommand{\blockwithtab}[3][]{
%  \begin{tikzpicture}
%    % Draw the box
%    \node[anchor=north west, draw=black!75!black, fill=white, rounded corners=1mm, text width=0.8\linewidth, inner sep=5pt] (box) {#3};
%    % Draw the tab
%    \node[anchor=south, draw=black!75!black, fill=white, rounded corners=1mm, text width=1cm, inner sep=2pt, yshift=2pt] at (box.north east) {#2};
%  \end{tikzpicture}
%}

% Kaushik's code listings
% Taken from https://www.overleaf.com/learn/latex/code_listing
\definecolor{codegreen}{rgb}{0,0.6,0}
\definecolor{codegray}{rgb}{0.5,0.5,0.5}
\definecolor{codepurple}{rgb}{0.58,0,0.82}
\definecolor{backcolour}{rgb}{0.95,0.95,0.92}
\lstdefinestyle{kkcodestyle}{
    language=python,
    backgroundcolor=\color{backcolour},
    commentstyle=\color{codegreen},
    keywordstyle=\color{magenta},
    numberstyle=\tiny\color{codegray},
    stringstyle=\color{codepurple},
    basicstyle=\ttfamily\footnotesize,
    breakatwhitespace=false,
    breaklines=true,
    captionpos=b,
    keepspaces=true,
    showspaces=false,
    showstringspaces=false,
    aboveskip=0pt,
    belowskip=0pt,
    showtabs=false,
    tabsize=2,
    escapeinside=@@
}

% Luke Olson's definition box
\newtcolorbox{lukedef}[1][Definition:]{
colback=white,
colbacktitle=white,
coltitle=IllinoisOrange,
colframe=IllinoisBlue,
boxrule=1pt,
titlerule=0pt,
%arc=15pt,
title={\strut#1},
}

% Matt's code listings
\definecolor{mintedlikecommentcolor}{rgb}{0.16,0.51,0.51}
\definecolor{mintedlikekeywordcolor}{rgb}{0,0.6,0}
\definecolor{mintedlikestringcolor}{rgb}{0.79,0,0}
\newcommand{\mintedlikebasicstyle}[1]{\ttfamily#1\linespread{4}}
\lstdefinestyle{mintedlike}{
    language=python,
    basicstyle=\mintedlikebasicstyle{\normalsize},
    commentstyle=\itshape\color{mintedlikecommentcolor},
    keywordstyle=\color{mintedlikekeywordcolor},
    stringstyle=\color{mintedlikestringcolor},
    columns=fullflexible,
    breakatwhitespace=false,
    breaklines=false,
    keepspaces=true,
    showspaces=false,
    showstringspaces=false,
    showtabs=false,
    tabsize=4,
    belowskip=-0.6\baselineskip,
    escapeinside=&&
}
% End MJS Additions


\newtcolorbox{codeblock2}[1][]{
  colback=white,
  colframe=black!75!black,
  width=0.8\linewidth,
  arc=1pt,
  outer arc=1pt,
  left=1pt,
  right=1pt,
  top=1pt,
  bottom=0pt,
  boxsep=0pt,
  #1
}

\newcommand{\codeblockwithtab}[2]{
  \begin{tikzpicture}
    % Draw the tab
    \node[draw, fill=white, rounded corners=1mm, anchor=south east] 
      at (0.8\linewidth, 0.5cm) {Source: \href{#1}{\texttt{#2}}};
    % Draw the code block
      \node[anchor=north west] at (0,0) {
        \begin{codeblock2}
         \lstinputlisting[style=kkcodestyle, basicstyle=\tiny, language=Python]{Figures/mtc/rhs_sample.py}
        \end{codeblock2}
      };
  \end{tikzpicture}
}
\newcommand{\snippetbox}[2]{
  \begin{tikzpicture}
    % Draw the box
    \node[draw=black!75!black, fill=white, rounded corners=1mm, text width=0.8\linewidth, inner sep=5pt, anchor=north west] (box) {
        \lstinputlisting[style=kkcodestyle, basicstyle=\tiny, language=Python]{#1}
    };
    
    % Draw the tab
    \draw[thin, rounded corners=1mm] (box.north east) -- ++(0,0.3) -- ++(-1,0) [rounded corners=0mm] -- ++(0,-0.3) -- cycle;
    \node[font=\tiny] at ([xshift=-0.5cm, yshift=1.5mm] box.north east) {#2};    % Draw the tab
    % \draw[thick, rounded corners=1mm] (box.north east) -- ++(0,0.5) -- ++(-1,0) [rounded corners=0mm] -- ++(0,-0.5) -- cycle;
    % \node[font=\tiny] at ([xshift=-0.5cm, yshift=2.25mm] box.north east) {#2};
  \end{tikzpicture}
}
