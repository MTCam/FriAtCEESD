\documentclass[aspectratio=169]{beamer}

\input{../ceesd-macros.tex}

%\newif\ifmovies
%\moviestrue
%\moviesfalse

%\newif\ifsubs
%\substrue
%\subsfalse

\newif\ifnotes
\notestrue
%\notesfalse


% MTC additions - ripped from Kaushik's KK 02 Additions

% Taken from https://www.overleaf.com/learn/latex/code_listing
\definecolor{codegreen}{rgb}{0,0.6,0}
\definecolor{codegray}{rgb}{0.5,0.5,0.5}
\definecolor{codepurple}{rgb}{0.58,0,0.82}
\definecolor{backcolour}{rgb}{0.95,0.95,0.92}
\lstdefinestyle{kkcodestyle}{
    language=python,
    backgroundcolor=\color{backcolour},
    commentstyle=\color{codegreen},
    keywordstyle=\color{magenta},
    numberstyle=\tiny\color{codegray},
    stringstyle=\color{codepurple},
    basicstyle=\ttfamily\footnotesize,
    breakatwhitespace=false,
    breaklines=true,
    captionpos=b,
    keepspaces=true,
    showspaces=false,
    showstringspaces=false,
    showtabs=false,
    tabsize=2,
    escapeinside=@@
}

% End MTC Additions

% LO start
\newtcolorbox{lukedef}[1][Definition:]{
colback=white,
colbacktitle=white,
coltitle=IllinoisOrange,
colframe=IllinoisBlue,
boxrule=1pt,
titlerule=0pt,
%arc=15pt,
title={\strut#1},
}
% LO end

% MJS Additions
\usepackage{listings}

\definecolor{mintedlikecommentcolor}{rgb}{0.16,0.51,0.51}
\definecolor{mintedlikekeywordcolor}{rgb}{0,0.6,0}
\definecolor{mintedlikestringcolor}{rgb}{0.79,0,0}
\newcommand{\mintedlikebasicstyle}[1]{\ttfamily#1\linespread{4}}
\lstdefinestyle{mintedlike}{
    language=python,
    basicstyle=\mintedlikebasicstyle{\normalsize},
    commentstyle=\itshape\color{mintedlikecommentcolor},
    keywordstyle=\color{mintedlikekeywordcolor},
    stringstyle=\color{mintedlikestringcolor},
    columns=fullflexible,
    breakatwhitespace=false,
    breaklines=false,
    keepspaces=true,
    showspaces=false,
    showstringspaces=false,
    showtabs=false,
    tabsize=4,
    belowskip=-0.6\baselineskip,
    escapeinside=&&
}
% End MJS Additions

\begin{document}


% %======================================================================
% \begin{frame}\frametitle{04 --- Simulations }
% \setcounter{framenumber}{0}


% \vspace*{0.3in}
% {
%   \tiny
%   \textbf{Details on full-system simulation status including a discussion of integration of the
%     necessary physics modules and scaling of the present code.}

%   You should address the following:
%   \begin{itemize}
%   \item The goals of this year’s full system demonstration(s), how it contributes to your
%     ultimate year-five predictive simulation, and the new results and insights into
%     predictiveness that were obtained.
%   \item Relative to the year-five prediction, what can and cannot be computed and predicted at
%     this point?
%   \item What are the key physics components still missing from your simulation as determined
%     by your roadmap and UQ process, and when do you expect to incorporate them?
%   \item Describe the scaling and performance obtained, including a discussion of any
%     limitations encountered.
%   \item How have you verified and validated your simulations?
%   \end{itemize}

% \textbf{Provide a slide outlining the major risks involved in reaching your predictive goal and a
%   mitigation plan for those risks.}
% }

% \end{frame}
% %======================================================================



%======================================================================
\begin{frame}\frametitle{}

\vspace*{0.2in}

\hspace*{0.0in}\textrm{{\huge\bfseries\color{myOrange} Fr@CEESD}}

\vspace*{0.2in}
\hrule
\begin{center}
\includegraphics[width=0.7\textwidth]{Figures/coverart-sim.pdf}
\end{center}
\hrule

\vspace*{0.1in}
\hfill\cPI{Mike Campbell} \rPI{(CS)}  % \\
%\ \hbox{}\hfill\cPI{Matt Smith} \rPI{(NCSA)}\\
%\ \hbox{}\hfill\cPI{Mike Campbell} \rPI{(NCSA)}\\
%\ \hbox{}\hfill\cPI{Mike Anderson} \rPI{(NCSA)}

\end{frame}
%======================================================================

%======================================================================
%\begin{frame}[fragile]\frametitle{Outline}
%  \begin{tabular}{m{8cm}m{6cm}}
%  \begin{itemize}
%    \setlength{\itemsep}{0.0in}
%    \item Y2/Y3 Simulation infrastructure\prj{\tiny}{M.~Smith}
%      \begin{itemize}
%        \item How-to guide to \mirgecom
%        \item Wall modeling and coupling
%        \item Distributed DAG discussion
%      \end{itemize}
%  \end{itemize}
%  &
%  \includegraphics[width=6cm]{./Figures/smith/wall_kappa-clip.png}
%  \\[0cm]
%  \begin{itemize}
%    \setlength{\itemsep}{0.0in}
%    \item \mirgecom{} Performance \prj{\tiny}{M.~Campbell}
%      \begin{itemize}
%        \item Verification overview: testing, coverage
%        \item Performance highlights: scaling, monitoring
%      \end{itemize}
%  \end{itemize}
%  &
%  \includegraphics[width=6cm]{./Figures/timing-clip.png}
%  \\
%  \begin{itemize}
%    \setlength{\itemsep}{0.2in}
%    \item Prediction results\prj{\tiny}{M.~Anderson}
%      \begin{itemize}
%        \item Simulation status for Y3
%        \item Parsl, workflow plans \prj{\tiny}{Doug Friedel}
%      \end{itemize}
%  \end{itemize}
%  &
%  \includegraphics[width=6cm]{./Figures/Noslip_isolator_clipped.png}
%  \\
%  \begin{itemize}
%    \setlength{\itemsep}{0.2in}
%    \item Summary and risks \prj{\tiny}{Freund, Olson}
%  \end{itemize}
%  &
%  \end{tabular}
%\end{frame}
%======================================================================


\input{thetalk-2023-10-13.tex}
\input{../ceesd-acknowledgement.tex}

\end{document}
